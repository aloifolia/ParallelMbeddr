
% ----------------------------------------
% This file contains shortcuts for custom
% commands with one or more arguments.
% Convention: UPPER CASE
% ----------------------------------------


% some colors for listings
\definecolor{dkgreen}{rgb}{0,0.6,0}
\definecolor{mauve}{rgb}{0.58,0,0.82}
\definecolor{silver}{rgb}{0.85,0.85,0.85}
\definecolor{silverlight}{rgb}{0.9,0.9,0.9}
\definecolor{noteyellow}{HTML}{FEE411}
\definecolor{lightnoteyellow}{HTML}{FEF18A}
\definecolor{orange}{HTML}{FF7F00}
\definecolor{todoblue}{HTML}{99CCFF}
\definecolor{respred}{HTML}{FF9966}


% - indicate when there are still things to do
%\newcommand{\TODO}[1]{}
\newcommand{\TODO}[1]{ {\colorbox{todoblue}{\begin{varwidth}[t]{0.7\textwidth}
				\textbf{TODO:}\ #1
			\end{varwidth}}\\}}
% - indicate missing code or images
\newcommand{\INSERT}[1]{\texttt{\large{\color{red}{INSERT: #1}}}}
% - use for code in prose text
\newcommand{\CODE}[1]{\texttt{\textbf{\colorbox{silver}{#1}}}}
% - use for code in prose text with red textcolor
\newcommand{\REDCODE}[1]{\textcolor{red}{\CODE{#1}}}
% - use to emphasize the following statement
%\newcommand{\NOTE}[2]{ }
\newcommand{\NOTE}[2]{ {\colorbox{lightnoteyellow}{\begin{varwidth}[t]{0.7\textwidth}
				\textbf{Kommentar von #1}\\
				#2
			\end{varwidth}}\\}}
			
\newcommand{\Bastian}[1]{\NOTE{Bastian}{#1}}
\newcommand{\Thomas}[1]{\NOTE{Thomas}{#1}}

% - use for signatures
% - Ja das ist nicht die schönste Art 3 Unterschriftenfelder nebeneinander zu bekommen mit Leeraum darüber.
\newcommand{\SIGNATURES}[3]{\begin{tabularx}{\textwidth}[b]{p{3cm} X p{3cm} X p{3cm}} \\\\\\\cline{1-1} \cline{3-3} \cline{5-5}

		#1 & & #2 & & #3\\\\\\
	\end{tabularx}}

% - use to mark responsibilities
\newcommand{\RESPONSIBLE}[1]{ {\colorbox{noteyellow}{\begin{varwidth}[t]{0.7\textwidth}
				\textbf{Verantwortlich f�r diesen Bereich}\\
#1
\end{varwidth}}\\}}

% - example header
\newcommand{\EX}[1][\empty]{\textbf{Example:} #1\par}
% - solution header
\newcommand{\SOL}[1][\empty]{\textbf{Solution:} #1\par}

% - example code
\newcommand{\EXC}[2][\empty]{
\EX[#1]
%$\blacktriangleright$ \texttt{\textbf{\colorbox{silverlight}{#2}}}}
$\blacktriangleright$ \texttt{\textbf{\fbox{\lstinline[language=Java, basicstyle=\bfseries, keepspaces]{#2}}}}}

% - inline code
\newcommand{\ICODE}[1]{
\fbox{\lstinline[language=Java, basicstyle=\bfseries, keepspaces]{#1}}}

% - introduce a new "Syntax" paragraph which contains syntax in bnf form
\newcommand{\SYNPAR}[1]{\paragraph{Syntax}\CODE{#1}\par}

% - use for programming languages (pl)
\newcommand{\PL}[1]{\textsc{#1}}
 
 % - a block of code, initiated with a header
 \lstnewenvironment{CBLOCK}[1][\empty]
  {\ifthenelse{\equal{#1}{\empty}}{}{\textbf{#1:}}
  \lstset{language=Java}}
  {}


% - a block of code, initiated with "Example:" + <optional description>
\lstnewenvironment{EXCBLOCK}[1][\empty]
 {\textbf{Example:} #1 \lstset{language=Java}}
 {}

% - a paragraph of code, initiated with "Definition:" + <optional description>
\lstnewenvironment{DEFCBLOCK}[1][\empty]
 {\paragraph{Definition} #1 \lstset{language=Java}}
 {}

