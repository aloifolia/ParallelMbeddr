\chapter{Optimization}
The previous chapter introduced the means to enable parallel execution of code via tasks and establish communication between tasks via shared resources. In order to make the communication thread-safe synchronization statements were introduced which provide synchronization contexts of atomic thread-safe blocks for the shared resources that they synchronize. For the purpose of both simplicity of the design and thread-safety of the user-code conservative restrictions were made: in the variability of the code and the scopes of synchronization contexts. While this strategy simplifies the construction of correct programs it may induce unnecessary serialization during program execution if potential data hazards will actually never manifest at runtime\cite{SpeculativeLockElision}. While the synchronization overhead reflects the optimization potential in the time-wise dimension there is also a space-dimension which originates from the way mutexes are used in the implementation. 

\section{Space Optimization}
As was briefly addressed the runtime memory consumption of the translated code is extended by the amount of memory that is occupied by the mutex maintenance. In addition to the obvious memory consumption of the handle and some internal management data the choice to make mutexes robust against recursive locks requires a counter field to be maintained throughout the lifetime of each mutex. This property impairs both the space and computation time overhead of the implementation unnecessarily in case shared resources are never recursively synchronized. If such cases can be detected (which they are, as will be shown in the next section) the generator could decide to declare the according mutexes as non-recursive. This optimization is left for future extensions of ParallelMbeddr. Another starting point would be the reduction of padding, i.e. unused data, that is automatically introduced by the compiler into the struct instances of shared resources. Padding is added into structs in order to retrieve data from memory more efficiently by aligning it along proper addresses \cite[p.~27]{MemoryAsAProgrammingConcept}. The amount of padding that is inserted depends on the difference of the byte sizes of the individual fields and the order of these fields \cite{MemoryAsAProgrammingConcept}. Since the ``art of C structure packing''\footnote{See http://www.catb.org/esr/structure-packing/ for details.} is no trivial task and space optimization is no primary concern of this paper, according work is left open for future research.



\section{Time Optimizations}
Various optimizations for lock-based synchronizations have been conceived. The general goal of all approaches is to minimize the overhead of synchronization measures. Among others this can be accomplished in two main dimensions. First, the amount of synchronization management (TODO), i.e. the time spent for acquiring and releasing locks, can be reduced by diminishing the number of performed locks. This technique is called \textit{lock elision}. Secondly, the waiting time of threads for the release of locks which they try to acquire, also known as \textit{lock contention}, can be diminished. While the classic approaches try to optimize the code at compile-time, increasing effort is performed towards optimizations which occur at run-time. The latter kind primarily resides in the domain of the transactional memory model \cite{SpeculativeLockElision}\cite{ARuntimeSystem}. Due to the complexity and overhead of such techniques this thesis focuses on compile-time optimizations.

The applicable measures for optimizations distinguish in their coverage of `perceived potential' and their performance. Superior techniques often require more comprehensive information, which, in turn, demands more sophisticated analyses. Such analyses, however, usually come with an increase of complexity. Typical candidates are \textit{pointer analysis} and \textit{data-flow analysis}.

\subsection{Pointer Analysis}
Pointer analysis, also known as \textit{points-to analysis}, tries to find the set of memory locations to which a pointer may point\cite{PointerAnalysisForStructuredParallelPrograms}. This information will be heavily used in the static analysis of this chapter. The following example illustrates the basic principle of pointer analyses. At the end of the if-else statement \CODE{p} may point to the location of either \CODE{v} or of \CODE{w}. Due to the copy-semantics of assignments in C, \CODE{q} will have the same value as \CODE{p} after the assignment in line 5. \CODE{p} can therefore be regarded as an \textit{alias} of \CODE{q}.
\begin{ccode}
int32 v, w;
int32* p;
if (condition)  p = &v;
else            p = &w; 
int32* q = p;
\end{ccode}
The quality of a pointer analysis is reflected by its precision. Two properties -- that are relevant for this work -- influence the precision: whether the analysis is \textit{flow-sensitive} or \textit{-insensitive} and whether it is \textit{inter-procedural} or \textit{intra-procedural}\footnote{The latter property is also called context-sensitivity \cite{CloningBasedContextSensitive}.} \cite{ProgramAnalysisAndSpecialization}. The first property distinguishes, whether the particular flow of data and with it the order of statements is taken account of the analysis. For an illustration the previous code listing is reconsidered. In a flow-insensitive analysis the set of locations for \CODE{p} would contain the locations of both \CODE{v} and \CODE{w} in either branch. A flow-sensitive analysis, on the other hand, would precisely assign \CODE{v} or \CODE{w} to the points-to set of (in the following also called \textit{alias set}) \CODE{p}. The second property distinguishes how precisely the calculation of points-to sets regards effects across functions, i.e. the context flow across function calls. The following example shall illustrate the difference between the two shapes. According to Andersen, an intra-procedural analysis would merge the calls of \CODE{bar()} so that the points-to set of \CODE{p} would contain both \CODE{v} and \CODE{w}. Likewise the sets of \CODE{vP} and \CODE{wP} would contain the same elements. On the hand, an inter-procedural analysis would distinguish the calls so that, in the end, the sets of \CODE{vP} and \CODE{wP} would contain exactly \CODE{v}, respectively \CODE{w}.
\begin{ccode}
void foo() {
  int32 v, w;
  int32* vP = bar(&v);
  int32* wP = bar(&w);
}
void bar(int32* p) {
  return p;
}
\end{ccode}
Sometimes it is necessary to not compute the aliases that a pointer can have in the course of a program run. Instead the set of locations that a pointer will point to on every possible path through the program might be needed. While the former is called a \textit{may point-to} analysis, the latter is called a \textit{must point-to} analysis \cite{ProgramAnalysisAndSpecialization}.
%The information delivered by a pointer analysis directly affects the quality of data-flow analyses which employ this information \cite{ContextInsensitiveAliasAnalysis}.
Currently, mbeddr does not provide any form of pointer analysis.

\subsection{Opportunities}
Static analysis offers a variety of optimization opportunities. These distinguish both in their optimization potential and the information needed for their realizations. Due to the similarity of shared resources to the synchronization concept of Java's monitors and the ongoing research in this area, the ideas were (mainly) influenced from the according literature \cite{StaticAnalysesForJava}\cite{JavaTheoryAndPractice}\cite{DoJava6Threading}. The opportunities are ordered in the way they are applied in ParallelMbeddr. The first three following paragraphs concern themselves with the elision of unnecessary locks and, hence, the computational overhead of synchronization management. The fourth focuses on the reduction of lock contention.

\subsubsection{Single-Task Locks}
The most obvious case, where locks for shared resources can be removed, is when synchronized data is only accessed by one task. This may happen for a limited amount of time, for instance in the time span from the declaration of a local variable of a shared resource to its first sharing with out tasks:
\begin{ccode}
void foo() {
  shared<int32> v;
  shared<int32>* vP = &v;
  init(vP);
  |task(vP)|;
}

void init(shared<int32>* var) {
  sync(var as varToSet) {
    varToSet->set(0);
  }
}

void task(shared<int32>* var) {
  sync(var as varToGet) {
    int32 val = varToGet->get;
  }
}
\end{ccode}
Whereas \CODE{varToSet} in \CODE{init()} need not be synchronized since it is not yet shared with another task, the according \CODE{varToGet} obviously needs to be synchronized inside \CODE{task()}.
Furthermore, shared resources may only be accessed by one task, at all. This can happen if the programmer does not work attentively. More importantly, the re-use of existing data structures and according functions for single-task data can cause the same effect. For instance, a thread-safe queue and functions to manage this queue could be re-used by the user for data that resides in only one task. It would then be helpful to distinguish the necessity of locks (i.e. sync resources) depending on the use of queue.

\subsubsection{Read-only Locks}
In ParallelMbeddr shared resources may actually never be set. For primitive data, e.g. of the type \CODE{shared<int32>} this should seldom be the case, if the user creates the code carefully. However, he might decide to use structs in order to pack independently synchronizable data:
\begin{ccode}
struct QueueContainer {
  shared<Queue>   queue;     // Queue is given as a black-box
  shared<boolean> isFull;
  shared<int32>   itemCount;
}

void foo() {
  shared<QueueContainer> queueC;
  shared<QueueContainer>* queueCP = &queueC;
  // ...
  sync(queueC) {
    sync(&queueC.get.isFull as isFull) {
      isFull->set(true);
    }
  }
}
\end{ccode}
Because of the semantics for variables of shared resources, \CODE{QueueContainer} can never be overwritten by another value. This is accomplished by mbeddr's non-typesystem rules. Therefore, any synchronization of \CODE{queueC} and any of its aliases is not necessary. However, the user still needs to pack the data into a shared resource in order to be able to safely share it with other tasks. Although the IDE could infer that \CODE{queueC} never needs to be synchronized the user still has to annotate the synchronization in line 11. This way the code need not be changed in case \CODE{QueueContainer} might eventually be equipped with non-shared data.\footnote{Of course, this } Nevertheless, the compiler should take care of eliminating locks for such data. In case functions are called with both read-only shared resources and written shared resources, the necessity of their according synchronization resources might depend on the respective calls (equivalent to single-task locks). This could for instance be the case for logging functions which only read the data of their arguments that shall be logged.

\subsubsection{Recursive Locks}
ParallelMbeddr does not prevent the user from acquiring locks for shared resources recursively. Instead, due to the scoping rules of synchronization resources and their contexts the programmer might be forced to synchronize shared resources recursively:
\begin{ccode}
void sort(shared<int32[1000]>* array, int32 start, int32 end)
int32 middle = ...
sort(array, start, middle);
sort(array, middle + 1, end);
sync(array) {
  // merge the sorted sub arrays
}
\end{ccode}
The example depicts a simplified version of a sort algorithm like merge sort. The function \CODE{sort()} recursively divides the array into smaller sub arrays and uses the sorted sub arrays to calculate a sorted version of the current array interval $[start, end]$. In the first call of \CODE{sort()} from outside, \CODE{array} might not yet be synchronized. In any other recursive call, however, \CODE{array} will definitely be already synchronized. Thus, at least in every sub-call of \CODE{sort()} the corresponding lock should be omitted.

\subsubsection{Lock Contention}
Besides the removal of locks, an important optimization opportunity is the reduction of lock contention. In order to accomplish this goal, the synchronization lists of synchronization statements should be shrunk to the absolute minimum. In the following, this technique is called \textit{lock narrowing}. For instance, the user might decide to apply coarse-grained synchronization by defining one big synchronization context inside a function:

\begin{ccode}
void calculate(shared<double>* result) {
  sync(result as myResult) {
    // do something that is expensive and unrelated to the argument
    double pi = calculatePi();
    // now use myArg
    myResult->set(pi);
    // again, something unrelated
    log(pi);
  }
}

double calculatePi() {...}
void log(double arg) {...}
\end{ccode}

The statements in 4 and 8 do not make any use of the synchronized argument \CODE{result}. Hence, it would be safe to move these statements out the synchronization context:

\begin{ccode}
void calculate(shared<double>* result) {
  double pi = calculatePi();
  sync(result as myResult) { myResult->set(pi); }
  log(pi);
}
\end{ccode}

One could argue that synchronization lists could even be split into multiple parts in order to separate statements whose evaluations access the current synchronization resources from those that do not:

\begin{minipage}{0.35\textwidth}
\begin{ccode}
void increment(shared<int32>* c) {
  int32 current, next;
  sync(c as myC) {
    current = myC->get;
    next = current + 1;
    myC->set(next);
  }
}
\end{ccode}
\end{minipage}
\begin{minipage}{0.2\textwidth}
\begin{center}
$\longrightarrow$

split
\end{center}
\end{minipage}
\begin{minipage}{0.4\textwidth}
\begin{ccode}
void increment(shared<int32>* c) {
  int32 current, next;
  sync(c as myC) { current = myC->get; }
  
  next = current + 1;

  sync(c as myC) { myC->set(next); }
}
\end{ccode}
\end{minipage}

Yet with such an aggressive strategy, the optimizer might split the code across data dependencies which were formerly reflected in the code by the scope of a synchronization statement. For instance, in the previous example the split does not take into consideration the data dependencies between \CODE{current}, \CODE{next} and \CODE{myC}. Therefore, when another call of \CODE{increment()} is executed in an interleaved fashion, the resulting code introduces data races for the shared resource that \CODE{c} points to.

\subsection{Performed Optimizations}
The optimizations that were performed in this work are direct realizations of the aforementioned leverage points. For the lack of supporting analyses in mbeddr when this thesis was written, the optimizations assume that all threads are executed simultaneously, in order to fit into the scope of this work. Thus, since mbeddr was missing a pointer-analysis, at first a simplified pointer analysis for the application in the optimization algorithms was conceived. In this analysis, differing from the usual approach, a separate alias set is computed for each local variable, argument and reference of both kinds.\footnote{The analysis is therefore \textit{inclusion-based}, i.e. alias sets may overlap \cite{CloningBasedContextSensitive}.} Additionally, divergent from the usual terminology, an alias in this context is a variable or argument of type \textit{shared<t>}. These differences result from the following facts
\begin{itemize}
\item only shared resources are considered;
\item shared resources are bound to memory locations as they may not be rewritten;
\item shared resources inside structured data like arrays and structs are not considered.
\end{itemize}

Also, it is assumed that everything which has a type \CODE{shared<t>} or \CODE{shared<t>*} (i.e. variables, arguments and expressions) has an alias set. For the former type this is clearly the location of the resource itself.

\subsubsection*{Pointer analysis}
As was mentioned, the analysis that was implemented in the course of this thesis makes several simplifications in order to fit into the scope of this work. The analysis is intra-procedural but flow-sensitive for the regarded language concepts. It can either compute \textit{must point-to} or \textit{may point-to} alias sets.
As a starting point, a simplified directed data-flow graph is constructed. The nodes of this graph consist of local variable declarations, arguments\footnote{Although generally arguments are the values that are passed to function parameters, in the course of this work no such distinction is made. Instead, the formal function parameters are called arguments and the values, which are passed to functions, are called argument values. This terminology is closer to the one established in mbeddr.} and references to either kind. For each reference $r$ to a local variable or an argument $x$ an arc $(x, r)$ is added to the graph. Furthermore, for each local variable $v$ of type \CODE{shared<u>} or \CODE{shared<u>*}, whose initialization expression is a reference $r'$ to a local variable or argument of the same type, an arc $(r', v)$ is added. The same is done for local variables of type \CODE{shared<u>*} whose initialization expressions reference local variables or arguments of type \CODE{shared<u>} by address. Equivalently to local variables and initialization expressions, arcs $(r, a)$ are added to the graph for according pairs of argument values $r$ and arguments $a$.
\TODO{Visualize}
The data-flow graph is used to perform a pointer analysis. The output of the analysis is a directed alias graph whose nodes comprise the nodes of the data-flow graph. Each arc $(u, v)$ of the alias graph connects a variable, argument or reference $u$ with a variable or argument $v$. In doing so, either $v$ is an alias for $u$ or $u$ refers to a variable $v2$ for which $v$ is an alias.\footnote{Thus, due to the inclusion of references, the alias graph is not one in the classical sense.} Trivially, loops $(u, u)$ for a variable or argument $u$ are contained, since every node is an alias for itself. With this initial setting the algorithm works as follows:
\begin{algorithmic}
\Function{aliases}{$g$, $\mathit{strict}$} \Comment{g is an inverse data-flow graph}
\ForAll{$v \gets$ \Call{Variables And Arguments}{$g$} of type \CODE{shared<t>}}
  \State \Call{add}{$a$, $(v, v)$} \Comment{a is the new alias graph}
\EndFor
\Repeat
  \State find some $(n, m) \gets$ \Call{arcs}{$g$} where not $a[n]$ contains all $a[m]$

  \Comment{in strict mode only must-aliases are considered $\Rightarrow$ all in-nodes must then have the same aliases}
  \If{$\mathit{strict}$ and $n$ is argument and there are $i, j \in g[n]$ where $a[i] \neq a[j]$}
    \State skip $(n, m)$
  \Else
    \State $a[n]$.\Call{add all}{$a[m]$}
    \If{$n$ is local variable}
      \ForAll{$r \gets$ \Call{Following References}{$n$} whose targets are in $a[n]$}
        \State $a[r]$.\Call{add all}{$a[n]$}
      \EndFor
    \EndIf
  \EndIf

\Until{no more changes possible}
\EndFunction
\end{algorithmic}
\textit{aliases} propagates alias information through the alias graph as long as any changes are possible. It repeatedly tries to find a node $n$ who does not contain all aliases of a node $m$ for which an arc $(m, n)$ resides in the original data-flow graph. In such a case $n$ gets connected with the missing aliases. If \textit{must point-to} aliases need to be calculated (which is currently necessary for the recursive-lock elision algorithm), all nodes $l$ for which an arc $(l, n)$ exists, must have equal aliases in order to be applicable for an alias augmentation of $n$. Due to the simplifications of the analysis, this can only be the case for value-to-argument arcs, since no other branches are considered. 

\subsubsection{Removal of Single-task Locks}
In order to remove synchronization resources of variables which are only used in one task, the aforementioned alias analysis is performed. It is fed with the inverse of a data-flow graph as it is delivered by the aforementioned function. 
\begin{algorithmic}
\Function{remove singles}{g, a, d} \Comment{g = inverse data-flow graph, a = aliases, d = additional data}
\ForAll{$v \gets d.\mathit{variables}$}
  \If{there is no $(n, m)$ in $g$ where $a[n]$ contains $v$ and $n$ and $m$ reside in different tasks}
    \State $\overline{c}$.\Call{add}{$v$} \Comment{$\overline{c}$ = single (clean) task variables}
  \EndIf
\EndFor
\State \Call{remove clean locks}{c, a, d}
\EndFunction
\end{algorithmic}
\textit{Remove Singles} gathers all variables which never leave their defining task contexts. This is accomplished by investigating for each variable $v$ all references whose alias set contains $v$. If any if these references leave the task context of the variable or argument $x$ that they reference -- which may happen if they reside in a task expression but $x$ does not --, then $v$ is no single task variable. The sought variables are thus gathered. The actual lock removal is accomplished by \CODE{Remove Clean Locks}, which is also used for the removal of read-only locks. The following pseudo-code depicts the general approach of the function:
\begin{algorithmic}
\Function{remove clean locks}{$\overline{c}$, $a$, $d$} \Comment{$\overline{c}$ = clean variables, $a$ = aliases, $d$ = additional data}
\ForAll{$s \gets$ \Call{sync resources}{data}}
  \If{$\overline{c}$ contains all $a[s.\mathit{expr}]$} \Comment{$s.\mathit{expr}$ evaluates to a shared resources or pointer thereof}
    \State $s$.\Call{remove}{}
  \Else
    \State $\mathit{f\_to\_\overline{s}}[\mathit{function\ of }s]$.\Call{add}{$s$} \Comment{$\mathit{f\_to\_\overline{s}}$ contains functions with partially clean sync resources}
  \EndIf
\EndFor

\ForAll{$(f, \bar{s} \gets \mathit{f\_to\_\overline{s}}$}
  \State \Call{try to inline}{$f$, $\bar{s}$, $\overline{c}$, $a$, $d$}
\EndFor

\EndFunction
\end{algorithmic}
Every synchronization resource $s$ is either directly removed or deferred. In case all aliases of the expression of $s$ are clean (e.g. they are all single task variables) $s$ can clearly be removed, since its synchronization is useless. Otherwise at least some of its aliases might be clean. In case of single-task locks, these aliases would ideally originate from an argument like in the following example:
\begin{ccode}
void shareXButNotY() {
  shared<int32> x;
  shared<int32>* xP = &x;
  shared<int32> y;
  |xP|;                    // important: for |x| or |&x|, x would not be shared but copied
  syncXOrY(&x);
  syncXOrY(&y);
}

void syncXOrY(shared<int32>* xOrY) {
  sync(xOrY as val) { val.set(0); }
}
\end{ccode}

In this example, \CODE{x} is shared, but \CODE{y} is not. Therefore, the set of clean variables $c$ in \CODE{Remove Clean Locks} would only contain \CODE{y}. On the other hand, the set of aliases $a[$\CODE{xOrY}$]$ would contain both variables, as they would be forwarded to \CODE{xOrY} via the calls \CODE{syncXOrY(\&x)} and \CODE{syncXOrY(\&y)} in the aliasing analysis.\footnote{This merge is caused by the inter-procedural property of the current alias analysis. In case of an intra-procedural pointer analysis the following work would be facilitated.} In this case the function \textit{Try To Inline} would distinguish the respective calls and learn that for \CODE{syncXOrY(\&y)} no synchronization is needed for \CODE{xOrY}. In such a case, the function could be inlined for the safe call and the clean synchronization resource could be removed. Alternatively, as it is done by \textit{Try To Inline}, the function can be copied and accordingly optimized:\footnote{Function copying instead of inlining is done since this approach also works for recursive functions.}
\begin{ccode}
  //...
  syncXOrY(&x);
  syncXOrY_1(&y);
}
void syncXOrY(shared<int32>* xOrY) {
  sync(xOrY as val) { val.set(0); }
}
void syncXOrY_1(shared<int32>* xOrY) {
  shared<int32>* val = xOrY;
  sync() { val.set(0); }               // the empty sync will be removed
}
\end{ccode}

However, if the aliases of the synchronization resource's expression $e$ are not received via paths to the arguments of the surrounding function, function inlining (or copying) will not help. This may for instance be the case if $e$ refers to a local variable of type \CODE{shared<t>} that resides in the same function. Another possibility is that it refers to a global variable whose value was set inside another function. In order to match the simplified pointer analysis, currently only synchronization resources are considered whose expressions directly refer to one of the arguments of the surrounding function (like in the previous example). The \textit{Try To Inline} function works as follows:

\begin{algorithmic}
\State \Comment{$f$ = function, $\overline{ds}$ = partially clean sync resources, $\overline{c}$ = clean variables, $a$ = aliases, $d$ = add. data}
\Function{Try To Inline}{$f$, $\overline{ds}$, $\overline{c}$, $a$, $d$}
\ForAll{$ds \gets \overline{ds}$} \Comment{for each call find the sync resources that are clean}
  \State $\overline{da}$ = $a[ds.\mathit{expression}]$ which is not in $\overline{c}$ \Comment{$\overline{da}$ = dirty aliases for $ds$}
  \ForAll{$l \gets$ \Call{Clean Calls For}{$ds$, $\overline{da}$, $f$, $a$}} \Comment{$l$ = clean calls for $ds$}
    \State $l\_to\_\overline{cs}[l]$.\Call{add}{$ds$} \Comment{$l\_to\_\overline{cs}$ = clean syncs for call $l$}
  \EndFor
\EndFor
\ForAll{$l$, $\overline{cs} \gets l\_to\_\overline{cs}$} \Comment{pack the calls by equal sets of clean sync resources}
  \State $\overline{cs}\_to\_\overline{l}[\overline{cs}]$.\Call{add}{$l$}
\EndFor
\ForAll{$\overline{cs}$, $\overline{l} \gets \overline{cs}\_to\_\overline{l}$} \Comment copy the function for calls of equal sets of clean sync resources
  \State \Call{Copy Function}{$f$, $\overline{cs}$, $\overline{l}$}
\EndFor
\EndFunction
\end{algorithmic}
\textit{Try To Inline} first considers all partially clean synchronization resources $\overline{ds}$, i.e. synchronization resources whose expression aliases are clean for at least one function call. For each $ds$ of these $\overline{ds}$ it uses the function \textit{Clean Calls For} mine the function calls $\overline{l}$, which do not need $ds$. This means that the $ds$ actually needs to get its aliases from one of the arguments of its function (otherwise function inlining would be useless). Furthermore, every call in $\overline{l}$ must not contain any of the dirty aliases of $ds$. Hence, they can only originate from some other call. In case of an inter-procedural pointer analysis this information would certainly be easier to gather. For each synchronization resource $ds$ with a non-empty set $\overline{l}$ for each call a mapping to $ds$ is established. These mappings are then used in order to cluster calls which have equal sets of clean synchronization resources. These clusters are then used by \textit{Copy Function} to generate optimized versions of the current functions. For the lack of valuable insight, the definitions of \textit{Clean Calls For} and \textit{Copy Function} are skipped.

\subsubsection{Removal of Read-only Locks}
Read-only locks are removed equivalently to single-task locks. It differs in the condition that it uses to determine whether locks for a specific variable may be removed: 
\begin{algorithmic}
\Function{remove readonlys}{g, a, d} \Comment{g = inverse data-flow graph, a = aliases, d = additional data}
\ForAll{$v \gets d.\mathit{variables}$}
  \If{$\exists\ e.\mathit{set(\_)}$ in $d.\mathit{sharedSets}$ where $a[e]$ contains $v$}
    \State skip $v$
  \ElsIf{$\exists\ \mathit{e.get}$ in $d.\mathit{sharedGets}$ where $a[e]$ contains $v$ and $\exists\ e' = \_$ where $e'$ contains $e$}
    \State skip $v$
  \EndIf
  \State $\overline{c}$.\Call{add}{$v$} \Comment{$\overline{c}$ = readonly (clean) task variables}
\EndFor
\State \Call{remove clean locks}{c, a, d}
\EndFunction
\end{algorithmic}
A few things should be noted for the last two optimizations. First, in the actual implementation, the gathering of `clean variables' is separated from the actual call of \textit{Remove Clean Locks}. Instead, variants of the two algorithms, called \textit{Get Singles} and \textit{Get Readonlys}, are used to first gather all variables for which locks can be removed. Only then \textit{Remove Clean Locks} is applied to the union of both variable sets. This way, a function may be copied only once, if it is called multiple times with (a) shared resources that are actually shared and written, (b) read-only shared resources and (c) single-task shared resources. The second thing to consider is that these removal functions are actually called as often as some optimization by their application is possible. Otherwise, optimizations might not be possible if an argument value is itself `partially clean' like \CODE{var} in the following example:
\begin{ccode}
void send() {
  shared<int32> a;
  shared<int32> b;          // b is never shared or written
  shared<int32>* aP = &a;
  |aP|;
  sync(a) { a.set(0); }
  
  forward(&a);
  forward(&b);
}

void forward(shared<int32>* var) {
  sync(var as myVar) { myVar.get; }
  receive(var);
}

void receive(shared<int32>* var) {
  sync(var as myVar) { myVar.get; }
}
\end{ccode} 
Since \CODE{b} is never shared or written, its synchronizations inside \CODE{forward()} and \CODE{receive()} are useless. \CODE{a} on the other hand must always be synchronized. In the first optimization round the values of the calls of \CODE{forward} can strictly be separated into `clean' values (\CODE{b}) and `dirty' values (\CODE{a}). Therefore, an optimized variant for \CODE{forward(\&b)} can be generated. On the other hand, there is only one call for \CODE{receive()} and the aliases of its value are currently merged, because the pointer-analysis is intra-procedural. Thus, another round of optimizations has to be applied where the two calls of the now separated functions \CODE{forward()} (for \CODE{a}) and \CODE{forward\_0()} (for \CODE{b}) can be distinguished. An intra-procedural pointer-analysis could directly deliver this information. The third point to consider is that function duplication does not work across functions which do not split the aliases of their arguments themselves, as it was done by \CODE{forward()}. If the synchronization statement was removed in this function, the algorithm would not detect that a duplication of \CODE{forward()} might help. In consequence, \CODE{receive()} would not be duplicated as well. For this reason, in the future cross-function duplications should also be made available. On the other hand, like inlining function duplication can lead to a `code explosion' if it is used too aggressively. Thus, it should be controllable by the programmer.

\subsubsection{Removal of Recursive Locks}
In contrast to the previous optimization techniques, in this section the property, which must be proven in order to be able to remove a lock, does not hold for entire variables. Instead, the recursiveness of a lock has to be shown separately for every synchronization resource. For the basic idea of a recursive-lock removal algorithm an arbitrary synchronization resource $r$ of an expression $e$ is considered. If can be shown that for all aliases of the pointer or shared resource, which $e$ evaluates to, $e$ must already be synchronized, then $r$ can be removed. In order to facilitate the analysis, the data-flow graph, which is necessarily used in such an algorithm, should be preprocessed in the following way. First, edges due to recursive function calls should be removed. The rationale behind this constraint is that recursive functions should not be able to synchronize their arguments on their own. In the following code listing an analysis \CODE{recursive()} could otherwise  detect that \CODE{value1} \CODE{value2} are already synchronized at the beginning of the function.
\begin{ccode}
void recursive(shared<int32>* value1, shared<int32>* value2) {
  sync(value2 as myValue2) {
    myValue2.set(2);
  }
  sync(value1 as myValue1) {
    myValue1.set(5);
    recursive(myValue1, myValue1);
  }
}
\end{ccode}

Additionally, if there is another call to \CODE{recursive()} from outside the cyclic call structure the algorithm need not first check whether any of the calls to \CODE{recursive()} originate from a recursive call path, in order to treat it differently. Furthermore, while in the previous example recursions would be easy to detect indirect recursions across multiple functions are more complicated to handle. For these reasons, the directed data-flow graph should be created by starting in the main function and avoiding edges that close cycles in the corresponding call graph. Secondly, the algorithm checks whether the expression of a synchronization resource is already synchronized edges between nodes of different task contexts in the data-flow graph should be ignored. This way, false recursive locks across tasks are omitted:
\begin{ccode}
void task1(shared<int32>* value) {
  sync(value as myValue) { value.set(1); 
    |task2(value)|.run                   // here, value is not synced anymore
  }
}

void task2(shared<int32>* value) {
  sync(value as myValue) { value.set(2); }
}
\end{ccode}
Since every task needs to synchronize its shared resources on its own, the synchronization context of \CODE{sync} in \CODE{task1} cannot be forwarded to \CODE{task2}. Additionally, it should be noted that synchronization contexts may not be forwarded via the wrapped values of shared resources. For instance, in the following example the reference to \CODE{myValue} is a reference to a synchronized value. Yet, this synchronization information may not be forwarded into \CODE{myContainer}. Otherwise a different task which accesses the value of this shared resource would not have to synchronize it any more.
\begin{ccode}
void task1(shared<int32>* var, shared<shared<int32>*>* container) {
  sync(var as myVar, container as myContainer) {
    myContainer->set(myVar);
  }
}
\end{ccode}
The construction of the alias graph should, however, only omit cyclic edges since aliases across tasks are completely valid. In the following course of the work two algorithms for the recursive-lock removal are presented.\footnote{The former algorithm was implemented for its simplicity in the light of the simplified scenarios. The second one on the other hand avoids the incremental flow of synchronization contexts through the data flow graph and might be better suited for more complicated scenarios with concepts like assignments, shared resources nested in structures and arrays and multidimensional pointers (e.g. shared<int32>**).} The first algorithm uses the alias graph to locally mark variable references inside synchronization statements whose aliases are all synchronized. It uses the data-flow graph to push the synchronization states forward across function contexts:

\begin{algorithmic}
\State \Comment{$g$ = data-flow graph (dfg) w/o recursions or task-crossings, $i$ = inv. dfg, $a$ = aliases, $d$ = add. data}
\Function{Remove Recursive Locks 1}{$g$, $i$, $a$, $d$}
\ForAll{$s \gets d.\mathit{syncResources}$} \Comment{spread the sync contexts locally}
  \ForAll{$r \gets$ \Call{References In}{$s$}}
    \If{$r \neq s.\mathit{expr}$ and \Call{Task}{$r$} $=$  \Call{Task}{$s$} and $a[s.\mathit{expr}] \supseteq a[r] \neq \emptyset$} 
      \State $\overline{s}.$\Call{Add}{$r$} \Comment{$\overline{s}$ = references to synchronized shared resources}
    \EndIf
  \EndFor
\EndFor
\Repeat \Comment {spread the sync contextes globally}
  \ForAll{$s \gets \overline{s}$}
    \If{there is some $n$ in $d[s]$ which is not in $\overline{s}$ and $\overline{s}$.\Call{ContainsAll}{$i[n]$}} \Comment{$n$ = next node}
      \State $\overline{s}.$\Call{Add}{$n$}
    \EndIf
  \EndFor
\Until{no more change is possible}
\ForAll{$s \gets d.\mathit{syncResources}$} \Comment{remove recursive locks}
  \If{$\overline{s}$.\Call{Contains}{$s.\mathit{expr}$}}
      \State remove $s$
  \EndIf
\EndFor
\EndFunction
\end{algorithmic}
After the synchronization contexts have been flowed through the data-flow graph, the function can determine whether the expression of any synchronization resource $s$ evaluates to an already synchronized shared resource. In such a case $s$ can safely be removed. It should be noted that the implementation of the algorithm is currently not able to remove recursive locks across functions reliably. This limitation results from the fact that the applied primitive pointer analysis works intra-procedural. In consequence, the aliases for a function argument would usually be merged for different calls of a function. For an illustration the following example is considered:
\begin{ccode}
void foo() {
  shared<int32> p1; 
  shared<int32>* p1P = &p1; 
  shared<int32> p2; 
  shared<int32>* p2P = &p2;
  |p1P|; 
  |p2P|;
  bar(p1P, p1P);
  bar(p2P, p1P);
}

void bar(shared<int32>* v1, shared<int32>* v2) { 
  sync(v2 as myV2, v1 as myV1) { 
    myV1->set(1); 
    myV2->set(2); 
  } 
}
\end{ccode}
Due to the merge, the alias set of \CODE{myV1} becomes $\{$\CODE{p1}, \CODE{p2}$\}$ and dominates the alias set $\{$\CODE{p1}$\}$ of \CODE{myV2}. Hence, in the recursive lock algorithm the synchronization resource \CODE{myV2} would removed. The optimized code would therefore become:
\begin{ccode}
...
void bar(shared<int32>* v1, shared<int32>* v2) { 
  shared<int32>* myV2 = v2; 
  shared<int32>* myV1 = v1;
  sync(myV1) { 
    myV1->set(1); 
    myV2->set(2); 
  } 
}
\end{ccode}
Thus, for both calls \CODE{bar(p1P, p1P)} and \CODE{bar(p2P, p1P)} only \CODE{myV1} would be synchronized. In consequence, in the second call only the shared resource of \CODE{p2} would be synchronized, although both \CODE{p2} and \CODE{p1} would be modified via \CODE{myV1->set(1)}, respectively \CODE{myV2->set(2)}. This problem is currently faced by applying a \textit{must point-to} pointer-analysis for this optimization technique. As a result, the alias sets in the example would become empty. Thus, the condition $a[s.\mathit{expr}] \supseteq a[r] \neq \emptyset$ in the algorithm requires that the alias set of a synchronization resource is not empty in order to be able to decide that it is superfluous. For testing purposes this restriction can be deactivated (which happens in the evaluation chapter). In the future, an inter-procedural pointer analysis should be applied in order to check for a synchronization resource whether for every call it is covered by another one. If that is the case, it may be removed. The non-emptyness condition can then be omitted. As in the previous algorithms it would further make sense to apply function inlining/copying in order to remove synchronization resources for arguments -- of a function $f$--, which are used recursively only for a strict subset of calls of $f$. 

Another problem results from the restricted range of expressions and types that are currently supported in the optimization algorithm. If for instance a shared resource is nested inside an array \CODE{a} of type \CODE{shared<int32>[100]} and accessed via an array-access expression \CODE{\&a[5]}, the optimizer is not able to identify the accessed shared resource. Thus, if this expression is used as an initialization for a local variable \CODE{v} of type \CODE{shared<int32>*}, the alias-set of \CODE{v} will become empty. The same property holds for structs. For this reason, the data-flow construction and alias-analysis can be optionally used with support for \textit{pseudo aliases}. A pseudo alias identifies a shared resource that is not directly supported by the algorithm. Thus, whenever a pointer to a shared resource is accessed via an array-access or struct-access expression and assigned to a local variable or an argument $x$, then a pseudo-alias (currently an empty local variable) is added to the alias-set of $x$. This way, the optimizer can generally also detect recursive locks for not supported data structures, like in the following example:
\begin{ccode}
struct Container { shared<int32> value; }
void foo() {
  Container c;
  // ... share c with some other task
  sync(&c.value as myValue) {
    myValue->set(0);
    bar(value);
  }
}

void bar(shared<int32>* value) {
  sync(value as myValue) { myValue->set(1); }
}
\end{ccode}
The alias set of the local variable \CODE{myValue}, which the named resource in line 5 is reduced to, will be added a pseudo-alias $l$. $l$ will be forwarded to \CODE{value} in \CODE{bar()} and reach the local variable \CODE{myValue} in the same function. Since the synchronization context of \CODE{myValue} in \CODE{foo()} will take the same path, the compiler will find that the synchronization of \CODE{myValue} in \CODE{bar()} can be removed. However, this approach is not able to detect the equality of pseudo-aliases if they result from different expressions. Therefore, more extensive data structure support should be aimed for in the future.

The second algorithm uses the alias graph and the call graph, which connects the main function with all other functions that are induced by the data-flow graph. It considers every synchronization resource $s$ exactly once by determining whether the respective expression $e$ is synchronized for all aliased shared resources that $e$ might evaluate to. Every alias $x$ must  be covered by a synchronization statement either in the same function $f$ that $s$ resides in or in one of the functions that lie on the paths from the main function $f$ in the call graph. Hence, either $s$ must have an ancestor synchronization statement in the abstract syntax tree of $f$ that has another synchronization statement $s'$, whose aliases contain $x$. Or on every path from $f$ to \textit{main} there must each be at least one call that itself is nested in a synchronization statement with a synchronization resource whose alias set contains $x$. If for a synchronization resource this property holds it can be removed. The algorithm expects an inter-procedural analysis. Hence, in order to handle synchronization resources with references to function arguments correctly, it checks the alias-coverage for every call of $f$ separately.

\begin{algorithmic}
\State \Comment{$a$ = aliases, $d$ = add. data, $\mathit{f\_to\_\overline{\overline{c}}}$ = function to main paths' calls}
\Function{Remove Recursive Locks 2}{$a$, $d$, $\mathit{f\_to\_\overline{\overline{c}}}$} 
\ForAll{$l \gets$ \Call{Calls Of Function Of}{$s$}}
  \ForAll{$s \gets d.\mathit{syncResources}$ in $l$}
    \State $r :=$ true \Comment{indicates whether $s$ can be removed}
    \ForAll{$x \gets a[s.\mathit{expr}]$}
      \State $r := r \land$ (\Call{Alias Covered In Sync}{$x$, $s$, $a$, $\mathit{f\_to\_\overline{\overline{c}}}$} $\lor$  \Call{Alias     Covered In Paths}{$x$, $s$, $a$,     $\mathit{f\_to\_\overline{\overline{c}}}$})
    \EndFor
    \If{$r$}
      \State remove $s$
    \EndIf
  \EndFor
\EndFor
\EndFunction
\end{algorithmic}


\begin{algorithmic}
\State \Comment{$x$ = alias, $n$ = current node, $a$ = all aliases, $\mathit{f\_to\_\overline{\overline{c}}}$ = function to main paths' calls}
\Function{Alias Covered In Sync}{$x$, $s$, $a$, $\mathit{f\_to\_\overline{\overline{c}}}$} 
\State \Return $\exists$ sync resource $s$ in \Call{surroundingSyncs}{$n$} $.$ \Call{Task}{$n$} $=$ \Call{Task}{$s$} and $a[s.\mathit{expr}]$.\Call{Contains}{$x$}
\EndFunction
\end{algorithmic}

\begin{algorithmic}
\State \Comment{$x$ = alias, $s$ = sync resource, $a$ = all aliases, $\mathit{f\_to\_\overline{\overline{c}}}$ = function to main paths' calls}
\Function{Alias Covered In Paths}{$x$, $s$, $a$, $\mathit{f\_to\_\overline{\overline{c}}}$}
\ForAll{$\overline{c} \gets \mathit{f\_to\_\overline{\overline{c}}}\ [$\Call{Function}{$s$}$]$} \Comment{consider all paths to the main function}
  \If{there is no $c$ in $\overline{c}$ with \Call{Alias Covered In Sync}{$x$, $c$, $a$, $\mathit{f\_to\_\overline{\overline{c}}}$}}
    \State \Return false
  \EndIf
\EndFor
\State \Return true
\EndFunction
\end{algorithmic}

\subsubsection{Narrowing Synchronization Statements}
Synchronization statements are narrowed by iteratively moving statements from the beginning of the statement lists to outside the synchronization statements. This process stops as soon as a statement is encountered for which the movement could introduce new data-races. Likewise statements at the end of each list are moved outside. In order to not interfere with the scopes of local variables, currently all moved statements of a synchronization statement $s$ and $s$ itself are nested inside a new statement block:

\begin{minipage}{0.3\textwidth}
\begin{ccode}
void foo(shared<int32>* var) {
  boolean b = true;
  sync(var as myVar) {
    boolean b = false;
    myVar->set(0);
  }
}
\end{ccode}
\end{minipage}
\begin{minipage}{0.2\textwidth}
\begin{center}
$\longrightarrow$

narrow
\end{center}
\end{minipage}
\begin{minipage}{0.3\textwidth}
\begin{ccode}
void foo(shared<int32>* var) {
  boolean b = true;
  {
    boolean b = false;
    sync(var as myVar) { myVar->set(0); }
  }
}
\end{ccode}
\end{minipage}

The crucial aspect of the narrowing algorithm is how to decide whether a statement may be moved. In case no other optimization has taken place before a trivial approach would be to check whether a first or last statement of the statement list of a synchronization statement $s$ contains any references to a variable that is also referenced by a synchronization resource of $s$.\footnote{References to named resources need not be considered because they are resolved before any optimization takes place.} Due to the locality of synchronization contexts from a user's perspective one could argue that this attempt would suffice. However, with the advent of the previous optimization techniques, this simple attempt could fail. If recursive locks are removed, the \CODE{.set} and \CODE{.get} expressions, which previously referred to according variables, then get their synchronization contexts from other synchronization resources. This entails for example that synchronization resources need no longer be synchronized locally (in the same function) but can be synchronized across functions. The simple approach would not regard such cases. Therefore, not the directly referred variables are compared, but the according aliases. The algorithm is depicted by the following pseudo-code:
\begin{algorithmic}
\State \Comment{$a$ = aliases, $d$ = add. data, $\mathit{\overline{ro}}$ = read-only aliases, $\mathit{\overline{st}}$ = single-task aliases}
\Function{Narrow Syncs}{$a$, $d$, $\mathit{\overline{ro}}$, $\mathit{\overline{st}}$} 
\ForAll{$s \gets d.\mathit{syncStatements}$}
  \While{there is $t$ = \Call{First Statement In}{$s$} where \Call{Can Be Shifted}{$t$}}
    \State $t$.\Call{Move To End Of}{$l_1$} \Comment{$l_1$ = list of first shifted statements}
  \EndWhile
  \While{there is $t$ = \Call{Last Statement In}{$s$} where \Call{Can Be Shifted}{$t$}}
    \State $t$.\Call{Move To Start Of}{$l_2$} \Comment{$l_2$ = list of first shifted statements}
  \EndWhile
  \If{either $l_1$ or $l_2$ is not empty}
    \State replace old $s$ with new block $\{\ l_1.\textit{members}, s, l_2.\textit{members}\ \}$
  \EndIf
\EndFor
\EndFunction
\end{algorithmic}
The helper function \textit{Can Be Shifted} gathers all expressions \CODE{e.get} and \CODE{e.set(\_)} that may be evaluated for a specific statement $t$. For this purpose it searches for such expressions: first in the AST of $t$ itself and then in the ASTs of all functions that might be called for the evaluation of $t$. For this purpose the complete function call branching of $t$\footnote{This branching comprises all directed subtrees (= aborescences) -- of called functions -- whose roots lie in the AST of $t$.} is investigated. A shift is declared as safe if it is safe for every $e$ in the found expressions \CODE{e.get} and \CODE{e.set(\_)}. In turn, for $e$ the shift is unsafe if its alias set is contained in one of the alias sets of the synchronization resources of the synchronization statement $s$. For this check only the aliases need to be considered that are not read-only or single-task shared resources.  The pseudo-code for 
\textit{Can Be Shifted} is:
\begin{algorithmic}
\State \Comment{$t$ = statement, $s$ = sync statement, $a$ = aliases, $\mathit{\overline{ro}}$ = read-only aliases, $\mathit{\overline{st}}$ = single-task aliases}
\Function{Can Be Shifted}{$t$, $s$, $a$ , $\mathit{\overline{ro}}$, $\mathit{\overline{st}}$} 
\ForAll{$e.set(\_)$ and $e.get$ in the AST of $t$ or in the AST of every function of \Call{Call Branching}{$t$}}
  \If{there is sync resource $r$ in $s$ with $a[r.\mathit{expr}]$.\Call{Contains All}{$a[e] \backslash (\mathit{\overline{ro}} \cup \mathit{\overline{st}})$}}
    \State \Return false
  \EndIf
\EndFor
\State \Return true
\EndFunction
\end{algorithmic}
Again, an intra-procedural analysis should be preferred in order to be able to optimize more aggressively. In the current implementation, however, the optimization does not use such precise information and instead optimizes quite conservatively. Therefore, it must instead check whether any alias of the variable reference $e$ is contained in the alias set of a synchronization resource $r$.
%mehr möglich unter Bezugnahme der Threadabläufe (wann läuft welcher thread?), aktuell: Jeder gleichzeitig; => erfordert komplexe analysen (dataflow/pointer)

%lock prediction (für vergleich): https://www.usenix.org/legacy/event/hotpar10/tech/full_papers/Lucia.pdf

%lock reservation (mutexe werden generell gemäß pattern gelockt => für voraussichtlichen nächsten thread kein lock notwendig, bei Ausnahme auf altes Muster zurückfallen): http://delivery.acm.org/10.1145/590000/582433/p130-kawachiya.pdf?ip=130.83.73.240&id=582433&acc=ACTIVE%20SERVICE&key=2BA2C432AB83DA15%2E24DDBA2ADC8180AB%2E4D4702B0C3E38B35%2E4D4702B0C3E38B35&CFID=339772033&CFTOKEN=34325505&__acm__=1408384933_9473458d99f7cd303c776ced0b81d5b7

%anmerken am anfang: wenn nicht anders erklärt, bedeutet thread-safety immer race-condition-frei im Bezug auf die reinen Datencontainer (shared resources), nicht auch in Bezug auf higher-level dependencies, siehe z.B. http://dl.acm.org/citation.cfm?id=965681