\subsubsection{Notation}
In the following sections, the concepts of the language of mbeddr are iteratively extended with new elements. For this purpose, based on the notation presented by Pierce \cite{TypesAndProgrammingLanguages}, the syntax of new concepts is described by notations like the following:
\begin{ccode}
e ::= ...| e'
\end{ccode}
This exemplary line extends the set of expressions of mbeddr by the expression \CODE{e'} which may contain arbitrary meta-variables like \CODE{e}. In order to fit into the type system of mbeddr, these concepts are equipped with \textit{type inference} rules, each of which, given a list of premises, derives the type of some concept in the conclusion:
\begin{align*}
\inference*[NewConcept]{\mathit{premise_1}, ..., \mathit{premise_n}}{\qquad\quad e' |- t' \qquad\quad} 
\end{align*}
The premises of the concepts are kept minimal but not exhaustive, which means that complex premises are given as informal explanations and are mostly explained in section \ref{safetyMeasures}. This way, the type inference rules are kept comprehensible and the explanations for the safety measures of ParallelMbeddr are kept closely together. Additionally, this separation resembles MPS' separation of type inference rules and non-typesystem rules.

The translation of a new concept in ParallelMbeddr to a base concept of mbeddr will be shown informally by listing the resulting code that is generated by the IDE after it was given some input code. To this end, if not expressed in the text, the symbol $\Longrightarrow$ will denote the translation of code $c_1$ to some other code $c_2$ in $c_1 \Longrightarrow c_2$. The translation of basic code will often entail the generation of additional code somewhere else in the program as a side effect, e.g. the translation of an expression to a function call may force the generator of mbeddr to generate the declaration of the called function first. These side effects will also be demonstrated by listings of the generated code.