\chapter{Design and Implementation}
In this chapter extension of mbeddr for parallel programming, called ParallelMbeddr, is introduced. To this end the new language features for C are explained each in terms of the design and the translation to plain mbeddr C code. Furthermore relevant implementation details are briefly depicted. A running example will help to illustrate the presented features.
\Bastian{TODO: explain running example}


\section{Tasks and Futures}
The basic parallelization element is a \textit{task}. It denotes a parallel unit of execution. Its named deliberately divergently from the prevalent parallelization terms(reference to basisc) in order to abstract from the concrete implementation which might change in the future.
\subsection{Design}
The syntax e of expressions in mbeddr is extended with

e ::= ... | |e|

When executed a task term yields a handle to an execution unit that when it is run executes the embraced expression and returns its value. If the type of the expression is void no value will be returned. The type of a task reflects this return value.

t ::= ... | Task<t>

Due to implementation reasons the embraced return type of a task must be either void or a pointer to the type of the embraced expression:

e :- void
-----------------
|e| :- Task<void>

e :- t, t != void
-----------------
|e| :- Task<t*>


\subsection{Translation}














\section{Tasks and Futures}
for all items mentioned in the language design and necessary intermediary MPS concepts:
\begin{itemize}
\item implemented structure
\item implemented typing rules
\item implemented generation rules
\end{itemize}
\section{Synchronization}
see above

