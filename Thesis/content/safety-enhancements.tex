\section{Safety measures}
\label{safetyMeasures}
So far the basic blocks that constitute parallel code execution and shared data synchronization, namely tasks, shared resources and synchronization thereof, were introduced. Still missing are most of the rules which ensure that only shared resources may be shared and that these can only be used in a sane way. The current section fills this gap by giving an informal overview of the rules that were implemented in ParallelMbeddr, categorized by their objectives. In the following paragraphs \CODE{t} denotes some arbitrary type.

\subsection{Avoidance of implicitly shared unprotected data}
Global variables can be accessed by any function for which they are visible. Therefore, they must have a type \CODE{shared<t>} in order to restrict any modifications of their values to synchronized contexts. This restriction can be too strong if a global variable is only accessed by exactly one thread. Nevertheless, in this paper, the conservative approach was chosen in order to establish a safe foundation. Future static code analysis should be leveraged to reliably detect the cases where restrictions can be loosened.
Another class of data that is inherently vulnerable for unsafe data sharing arises from static variables. In C, local variables that are declared static have a ``global lifetime'' \cite[p.~439]{ProgrammingInC}, which means that as with global variables, the addresses of their allocated memory does not change. Thus, they keep their values from one function call to the next. The main difference between static local variables and global variables is the respective visibility. Consequently, static variables must have a type \CODE{shared<t>} as well.
Finally, a base type \CODE{t} of a shared type may never be a pointer type with a base type other than a shared type. Otherwise the value of a shared resource would point to data that is not synchronized and would enable unprotected inter-task communication. For instance, in the following example the functions \CODE{foo()} and \CODE{bar()} do not block one another since they synchronize over different shared resources. Nevertheless they both write to the same location in memory, which causes a data race.
\begin{ccode}
// global variables:
shared<int32*> v1;
shared<int32*> v2;

int32 main(int32 argc, string[] argv) {
  int32 sharedValue;
  sync(v1, v2) {
    v1.set(&sharedValue);
    v2.set(&sharedValue);
  }
  |foo()|.run;
  |bar()|.run;
}

foo() {
  sync(v1) { *v1.get = 0; }
}
bar() {
  sync(v2) { *v2.get = 1; }
}
\end{ccode}

\subsection{Copying pointers to unshared data into tasks}
The pass-by-value semantics of C generally already ensure that any local data which is refereed from within a task expression is safely copied into the task. On execution, the task will not access the original data, but a copy thereof. On the other hand, this approach becomes unsafe as soon as local variables are copied whose values a plain pointers (pointers to something else than shared resources). When such a copied pointer is used inside a task to access a pointed-to memory location in an unsynchronized manner, it accesses data that might simultaneously be accessed by another task, e.g. the task by which this task was created, which knows the address of the data. To avoid this behavior, every pointer that might be copied into a task by accessing a local variable or a function argument from within a task expression must point to a shared resource, i.e. must be of type \CODE{shared<t>*}. It has to be noted that this does not only hold for the variables themselves, but also for nested fields of struct instances and array elements. Furthermore, arrays must not be copied into tasks unless they are wrapped in a struct field. Due to the internal treatment of pointers in C (see section \ref{cBasics}), the access to an array holding local variable would cause a copy of the address of the array into the task as a pointer. In consequence, references of local variables and arguments with type \CODE{t[]...[]} inside task expressions are not allowed. On the other hand, it is safe to have a struct with a an array field be copied into a task. In contrast to the former case the array would then be entirely copied along its surrounding struct instance.

\subsection{Unsynchronized access to synchronizable data}
As was already mentioned in section \ref*{sharedMemory} the value of a shared resource can only be accessed (retrieved or rewritten) from within a proper synchronization context. This approach ensures that no write to shared data invalidates any other write or read of the data. The according rule is that an expression \CODE{e.get} or \CODE{e.set} is only allowed if \CODE{e} is either a reference to a named resource in scope, i.e. a shared resource which is synchronized in a surrounding synchronization statement and bound to a new name; or if \CODE{e} is a reference to a variable with a shared resource as value which is also refereed to by a synchronization reference of a surrounding synchronization statement. By this restriction the following code would trigger an error message in ParallelMbeddr:
\begin{ccode}
shared<shared<int32>> v;
sync(v) {
  sync(v.get) {
    // error: e.get seems to be unsynchronized
    e.get.set(0);
  }
}
\end{ccode}
Although the previous code would not produce any synchronization gap, ParallelMbeddr does not recognize this since the expression \CODE{e.get} of \CODE{.set} does not refer to a named resource or a variable with a synchronized shared resource. Instead, for exactly this purpose, named resources were implemented which allow to rewrite the code in the following valid manner:
\begin{ccode}
shared<shared<int32>> v;
sync(v) {
  sync(&(v.get) as w) {
    w->set(0);
  }
}
\end{ccode}
The reasoning behind this was to simplify the implementation of the safety checking analysis. Again, the chosen approach can in certain cases be overly conservative. If write conflicts can never happen for a shared resource and, in consequence, data races thereof are impossible, it would be safe to access the variable outside any synchronization context despite the error message that is generated by the IDE. Moreover, by the applied lexical scoping ParallelMbeddr is not able to detect whether a shared resource is recursively synchronized across multiple function calls:
\begin{ccode}
  shared<int32> v;
  sync(v) { foo(&v); }
  ...

void foo(shared<int32>* v) {
  sync(v) { v->set(0); }
}
\end{ccode}
Both problems should be addressed by static analysis in order to (partially) detect such cases. The second problem could further be addressed by the introduction of a \textit{synced} type for shared resources that are already synchronized in the current scope.
\Bastian{Maybe move to future work chapter.}

\subsection{Address leakage of shared resource values}
In order to restrict any writes or reads of the values of shared resources to synchronization contexts, it is crucial to not leak the memory addresses of these values outside the protected synchronization context from where they could be accessed via the address operator (\CODE{\&}). The measures to keep the addresses encapsulated constrain the use of the address operator and the use of arrays: The first rule forbids any expressions \CODE{\&e} where \CODE{e} contains an expression sub path \CODE{eSub.get} and \CODE{e} does not evaluate to a shared resource. The latter condition allows the programmer to get the address of an encapsulated shared resource, which is unproblematic since shared resources may not be overwritten as is explained in the next section \ref{overwritingSharedResources}. The second rule states that an expression \CODE{e} of some array type contains an expression sub path \CODE{eSub.get} and the parent of \CODE{e} does not access a specific element of \CODE{e} is not allowed. Thus, any access to an (multidimensional) array that is encapsulated in a shared resource must actually be extended to an access of the innermost element of the array. Otherwise it would be possible to assign the array itself or, in the case of a multidimensional array, take an element of the array which itself is an array and assign it to an unprotected pointer variable. Hence, the address of the array or of an element of this array would be leaked. For instance in the following example, ParallelMbeddr would complain about an address leakage of the first element of the array wrapped inside \CODE{v}:
\begin{ccode}
shared<int32[5][10]> v;
int32* pointer;
// address leakage!
sync(v) { pointer = v.get[0]; }
\end{ccode}

\subsection{Overwriting shared resources}
\label{overwritingSharedResources}
Shared resources may never be overwritten. The reason for this regulation results from the following consideration. If a shared resource \CODE{r} shall be overwritten, e.g. by a direct assignment or by a \CODE{set} if \CODE{r} is nested inside another shared resource, it must be synchronized first since the overwriting could overlap with another access to \CODE{r} from within another task. Before the resource is rewritten, the mutex of \CODE{r} must be destroyed in order to prevent memory leaks. Furthermore, after the rewriting is done, the newly created mutex for \CODE{r} must be initialized prior to any usage. Hence, in the time between the destruction and the re-initialization, the mutex of \CODE{r} (respectively, \CODE{r'} after the rewrite is done, because the variable will refer to a new value) cannot be accessed in the synchronization attempt of any other simultaneously executed synchronization statement. Any such task would thus have to be blocked, which would complicate the compiler and decrease the performance of the executed code without offering any worthy advantage. In addition to this problem, the overwriting of a shared resource of a struct instance that itself contains a shared resource field \CODE{f} would invalidate any pointer \CODE{p} to \CODE{f}. \CODE{p} could therefore not be used anymore afterwards, which on the hand complies with the usual C semantics, but does not fit the safety-first approach of ParallelMbeddr.
Nevertheless, it is safe to copy a shared resource into the memory of a local variable declaration or of a function argument, i.e. a pass of a shared resource to a function or an initialization of a newly created local variable with a shared resource is valid. In these cases, the shared resources can only be used after their initializations with shared resource copies.
The safety enforcing rules are as follows: A variable that refers to a shared resource or to a value that contains a nested shared resource may not be assigned a new value (the initialization of a declaration is not a classical assignment). Additionally an expression \CODE{e.set(e')} is not allowed if \CODE{e'} is a shared resource or contains a shared resource, because \CODE{e.set(e')} is translated into an assignment (see section \ref{sharedTypesTranslation}).