%\newpage
%\section{Eidesstattliche Erklaerung (für Diplomarbeiten)}% In Seminararbeiten ist die Eidesstattliche Erklärung zu entfernen bzw. auszukommentieren. Der hier nachfolgende Text ist aus dem Merkblatt für DIPLOMARBEITEN. Bei der Anfertigung einer BACHELORARBEIT muss der entsprechende Text aus dem Merkblatt für Bachelorarbeiten eingesetzt werden!!
%\addcontentsline{toc}{section}{Eidesstattliche Erklaerung}%\addtocontents{toc}{\vfill}
%Ich erkläre an Eides Statt, dass ich meine Diplomarbeit mit dem Titel [\emph{Titel der Arbeit}] selbständig und ohne Benutzung anderer als der angegebenen Hilfsmittel angefertigt habe und dass ich alle Stellen, die ich wörtlich oder sinngemäß aus Veröffentlichungen entnommen habe, als solche kenntlich gemacht habe. Die Arbeit hat bisher in gleicher oder ähnlicher Form oder auszugsweise noch keiner Prüfungsbehörde vorgelegen.\\

%Ich versichere, dass die eingereichte schriftliche Fassung der auf dem beigefügten Medium gespeicherten Fassung entspricht. (Dieser Zusatz ist nur erforderlich, wenn die Abschlussarbeit auch als Datei abgegeben worden ist.)\\\\\\
%\noindent Kiel, den XX.XX.201X
%\begin{flushright}
%$\overline{~~~~~~~~~\mbox{(Name des Kandidaten)}~~~~~~~~~}$
%\end{flushright}